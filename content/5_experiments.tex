\chapter{Experiments}\label{ch:experiments}

% ==============
% INITIAL EXPERIMENTS
% ==============

\section{Initial Experiments}\label{sec:initial-experiments}

% TODO insert real data
\begin{table}[h]
\centering
\begin{tabular}{c|l|clll}
\textbf{Variant}              & \textbf{Training Steps}                                                                   & \textbf{Metric} & \textbf{Pre} & \textbf{Rec} & \textbf{F} \\ \hline
\multirow{3}{*}{\textbf{AAA}} & \multirow{3}{*}{\begin{tabular}[c]{@{}l@{}}8625 steps\\ (approx. 141 epochs)\end{tabular}} & \textbf{R-1}    & 38.49       & 29.52       & 31.92     \\
                              &                                                                                            & \textbf{R-2}    & 17.18       & 13.51       & 14.28     \\
                              &                                                                                            & \textbf{R-L}    & 34.89       & 27.01       & 26.87     \\ \hline
\multirow{3}{*}{\textbf{AAI}} & \multirow{3}{*}{\begin{tabular}[c]{@{}l@{}}1234 steps\\ (approx. 12 epochs)\end{tabular}}  & \textbf{R-1}    & TODO         & TODO         & TODO       \\
                              &                                                                                            & \textbf{R-2}    & TODO         & TODO         & TODO       \\
                              &                                                                                            & \textbf{R-L}    & TODO         & TODO         & TODO       \\ \hline
\multirow{3}{*}{\textbf{IAA}} & \multirow{3}{*}{\begin{tabular}[c]{@{}l@{}}1234 steps\\ (approx. 12 epochs)\end{tabular}}  & \textbf{R-1}    & TODO         & TODO         & TODO       \\
                              &                                                                                            & \textbf{R-2}    & TODO         & TODO         & TODO       \\
                              &                                                                                            & \textbf{R-L}    & TODO         & TODO         & TODO       \\ \hline
\multirow{3}{*}{\textbf{III}} & \multirow{3}{*}{\begin{tabular}[c]{@{}l@{}}13350 steps\\ (approx. 351 epochs)\end{tabular}}  & \textbf{R-1}    & 18.48         & 14.64         & 15.55       \\
                              &                                                                                            & \textbf{R-2}    & 3.55         & 2.93         & 2.99       \\
                              &                                                                                            & \textbf{R-L}    & 15.75         & 12.44         & 12.23       \\ \hline
\multirow{3}{*}{\textbf{CCC}} & \multirow{3}{*}{\begin{tabular}[c]{@{}l@{}}1234 steps\\ (approx. 12 epochs)\end{tabular}}  & \textbf{R-1}    & TODO         & TODO         & TODO       \\
                              &                                                                                            & \textbf{R-2}    & TODO         & TODO         & TODO       \\
                              &                                                                                            & \textbf{R-L}    & TODO         & TODO         & TODO       \\ \hline
\end{tabular}
\caption{Rouge scores of the initial experiments}
\label{tab:initial-experiment-rouge}
\end{table}

For the first experiments, the network is trained to summarize $n$ dialogue acts to $1$ sentence of a meeting's abstract.
Results for the following combinations are reported:

\begin{itemize}
\item Trained on the AMI train data set, validated on the AMI eval data set, tested on the AMI test data set (\textbf{AAA})
\item Trained on the AMI train data set, validated on the AMI eval data set, tested on the ICSI corpus (\textbf{AAI})
\item Trained on the ICSI corpus, validated on the AMI eval data set, tested on the AMI test data set (\textbf{IAA})
\item Trained, validated and tested on the ICSI corpus, using a 70/15/15 split (\textbf{III})
\item Trained, validated and tested on a combined AMI and ICSI corpus, using a 70/15/15 split (\textbf{CCC})
\end{itemize}

All the reported results were calculated on the test set.

Training took on average less than one hour on a single RTX 2080 Ti GPU.
The achieved ROUGE scores are shown in \cref{tab:initial-experiment-rouge}.
While the results for AAA, III, CCC are good, the network failed to generalize enough to obtain useful results for the two cross-corpus variations AAI and IAA.

\begin{figure}[h]
\begin{lstlisting}[numbers=none]
DA: [CLS] and lea l delivered standard with a fruity colour, but
    not too not too much. but, the mai i think th the standard
    must be some kind of attractive flashy colours. so i think it
    will be a better idea to have some flashy fruity colours as as
    a standard, and for the people who really want a more
    sophisticated, more traditional look, they're willing to pay
    that. [SEP]
HS: the remote will come in fruity colours as standard with the
    option of buying different exchangeable covers.
GS: the remote will be made of fruity colors.
------------------------------------------------------------------
DA: [CLS] that's that's what i'm gonna write b between now. i'm
    going to finish my end report. [SEP]
HS: the project manager will finish the final report.
MS: the project manager will create a final report
\end{lstlisting}
\caption{Comparison between human summary (HS) and machine generated summary (GS) of dialogue acts (DA) for variant AAA.}
\label{fig:initial-experiment-example}
\end{figure}

\Cref{fig:initial-experiment-example} shows two hand-picked examples of generated summaries, together with the input and the human summary.
It is easily visible, that the network tends to overfit the data, as the generated summary contains information that is not part of the actual input, but can only be guessed if the settings of AMI's scenario meeting is taken into account \eg that the final report is always written by the project manager.
This explains the bad results for the AAI and IAA variations, as the network tries to apply the setting of the AMI corpus to the ICSI corpus and vice versa.
It is also a strong indicator that more training data is necessary for the network to learn to generalize, especially considering the very specific context of the AMI scenario meetings.

However, the network is able to correctly interpret the context of a sentence and is -- at least to a certain degree -- able to generate new sentences.
The generated summary from the first example is not present in the training data at all, but constructed from multiple different sentences.
The generated summary from the second example is identically to a human summary that was used for training, but for another input.
This human summary with the matching dialogue act in the training data set is shown in \cref{fig:initial-experiment-training-example}.
Unfortunately, the second example is the much more common case and the network does not generate new sentences very often, but only copies them from the training examples.
By reducing the amount of training steps, the network tends to generate own sentences more often, but at the cost of lower scores.

\begin{figure}[h]
\begin{lstlisting}[numbers=none]
DA: because I have to write the final report now.
HS: The project manager will create a final report
\end{lstlisting}
\caption{The training example, which summary the network copied in the second example of \cref{fig:initial-experiment-example}.}
\label{fig:initial-experiment-training-example}
\end{figure}

% ================
% EXTENDED EXPERIMENTS
% ================

\section{Extended Experiments}

\begin{figure}[h]
\begin{lstlisting}[numbers=none]
the project manager gave an introduction to the goal of the
project, to create a trendy yet user-friendly remote. she
presented a long-range agenda for the whole project. the group
introduced themselves to each other and practiced with the meeting
room tools by drawing on the board. the project manager presented
the project budget, the projected price point, and the projected
profit aim for the project. then the group began a discussion
about their own experiences with remote controls to generate
initial design ideas for making the product user-friendly. they
discussed grouping features into a menu and adding an lcd display.
they also discussed the look of various materials that may be used 
n the design, in keeping with the company's goal to create
fashionable electronics. 
------------------------------------------------------------------
the group discussed making the remote universally compatible and
ergonomic design. the project manager states that the remote needs
to be original, trendy, and user - friendly. the team members then
participated in an exercise in which they drew their favorite
animals. the project manager discussed the financial goals of the
project, including the projected profit aim and price point for
the device. the team then discussed their experiences using
remotes in the past and what features to consider implementing in
the remote. the project manager closes the meeting. the marketing
expert discussed his findings from trend watching reports. the
project manager closed the meeting. 
\end{lstlisting}
\caption{Comparison of human summary (top) with machine generated summary (bottom).}
\label{fig:extended-experiment-example}
\end{figure}

% TODO Insert real values
\begin{table}[h]
\centering
\begin{tabular}{@{}clll@{}}
\toprule
\textbf{Metric} & \multicolumn{1}{c}{\textbf{Pre}} & \multicolumn{1}{c}{\textbf{Rec}} & \multicolumn{1}{c}{\textbf{F}} \\ \midrule
\textbf{R-1}    & 36.65                           & 32.60                           & 33.54                         \\
\textbf{R-2}    & 13.68                           & 13.02                           & 12.90                         \\
\textbf{R-L}    & 33.88                           & 30.14                           & 29.44                         \\ \bottomrule
\end{tabular}
\caption{Rouge scores of the extended experiments}
\label{tab:extended-experiment-rouge}
\end{table}

For the extended experiments, the topic segmentation described in \cref{ssec:ami-annotations} is used to split meeting transcripts into smaller pieces that are used as inputs for the same network trained in \cref{sec:initial-experiments}.
This results in one summary sentence for each topic of a meeting.
By concatenating all these summary sentences of one meeting, one can get a summary of the entire meeting.
For this experiment, only the AMI Meeting Corpus is used with the standard data split described in \cref{ssec:ami-segmentation-of-the-corpus}, as there is no manual topic segmentation available for the ICSI Meeting Corpus. % TODO Maybe use the automatic segmentation with the automatic transcript?

An example for a generated summary is shown in \cref{fig:extended-experiment-example}.
The achieved ROUGE scores are shown in \cref{tab:extended-experiment-rouge}.

\paragraph{Comparison With Other Approaches}

\begin{table}[]
\centering
\begin{tabular}{@{}lcc@{}}
\toprule
                       & \textbf{R-1}     & \textbf{R-2}     \\ \midrule
\textbf{Neural (Ours)} & 33.54                & \underline{\textbf{12.90}} \\
\textbf{Ref-A}         & \underline{\textbf{37.86}} & 7.84                 \\
\textbf{Ref-B}         & 30.6                 & 6.8                  \\
\textbf{Ref-C}         & 31.5                 & 6.7                  \\
\textbf{Ref-D}         & 28.7                 & 4.2                  \\
\textbf{Baseline}      & 28.32                & 8.48                 \\ \bottomrule
\end{tabular}
\caption{Comparison with different approaches for abstractive meeting summarization and a baseline}
\label{tab:comparison}
\end{table}

Unfortunately, the used metrics to report results are very different and many of them are only reported for the AMI meeting corpus.
Some researchers only report the ROUGE-1 and ROUGE-2 scores using the F1 measure, while others also report ROUGE-L or ROUGE-SU4 scores with recall, precision and F1 measures.
For this reason, this thesis will only compare ROUGE-1 and ROUGE-2 scores using the F1 measure for the AMI corpus, as these scores are reported for nearly every approach.

The results of this work's neural approach are compared with the following approaches for abstractive meeting summarization:
\begin{itemize}
\item The results from \cite{shang-etal-2018-unsupervised}, using their best performing approach on AMI (\textbf{Ref-A}).
\item The results from \cite{oya-etal-2014-template}, using their 15 segment (\textbf{Ref-B}) and 20 segment system (\textbf{Ref-C}).
\item The results from \cite{mehdad-etal-2013-abstractive}, using their full model (\textbf{Ref-D}).
\end{itemize}

These scores are listed in \cref{tab:comparison} together with the neural network's scores and a baseline that is described later in this section.
When just comparing the plain ROUGE scores, the neural approach achieves competitive results for ROUGE-1 scores and surpasses the ROUGE-2 scores of any of the other approaches.
However, this comparison is not fair, as the other approaches do only use the dialogue acts or transcripts to generate their summaries.
While this is also true for the model of this work, it has access to the abstractive summaries of the training set during its training.
Because all of AMI's scenario meetings have a very similar structure, the abstracts in the training data tend to be very similar to the abstracts in the test data set.
This gives the network an unfair advantage over the other approaches and makes the results hardly comparable.

\begin{table}[h]
\centering
\begin{tabular}{@{}clll@{}}
\toprule
\textbf{Metric} & \multicolumn{1}{c}{\textbf{Pre}} & \multicolumn{1}{c}{\textbf{Rec}} & \multicolumn{1}{c}{\textbf{F}} \\ \midrule
\textbf{R-1}    & 26.25                           & 32.64                           & 28.32                         \\
\textbf{R-2}    & 07.88                           & 09.92                           & 08.48                         \\
\textbf{R-L}    & 23.84                           & 29.57                           & 24.40                         \\ \bottomrule
\end{tabular}
\caption{Rouge scores of random baseline}
\label{tab:extended-experiment-rouge-random}
\end{table}

To still have a baseline that results can be compared to, one can just randomly pick a summary sentence from the training data for every topic.
If the network is able to understand the meaning of its input, it should outperform this baseline.
This baseline is also included in the comparison table \cref{tab:comparison} and all scores are also shown in \cref{tab:extended-experiment-rouge-random}.
When comparing the network's scores with this baseline, the results of the network indeed outperform the baseline by a significant margin.
The fact, that even this simple baseline achieves higher ROUGE-2 scores than any of the other unsupervised approaches is a clear indicator, that the access to the abstracts of the training data is indeed an unfair advantage.

\paragraph{Issues with the generated summary}

The generated summaries come with some issues, that are not visible when just looking at the ROUGE scores.
One example are duplicate sentences, or sentences that carry the same meaning, like the two sentences "the project manager closes the meeting." and "the project manager closed the meeting." that are both part of the same summary shown in \cref{fig:extended-experiment-example}.
While this issue could be solved by some clever filtering of duplicates, a more severe issue is wrong information in the summaries.
E.g. the sentence "the marketing expert discussed his findings from trend watching reports" is incorrect for the meeting from \cref{fig:extended-experiment-example}.