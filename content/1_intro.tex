\chapter{Introduction}\label{ch:Introduction}

% Motivation

Transfer Learning is known to improve results on a significant number and variation of machine learning tasks by applying learned knowledge from one task to another task.
For instance, Google's recently announced language-representation model, Bidirectional Encoder Representations from Transformers (BERT), broke records for many natural language processing benchmarks \cite[p.~5--7]{devlin2018bert} such as  SQuAD v1.1 \cite{rajpurkar-etal-2016-squad} and GLUE \cite{1804.07461}.
There have also been attempts to apply BERT to the problem of abstractive summarization(\cite{1902.09243} and \cite{1908.08345}); however, they are usually trained and tested on news articles.
Generating abstractive summaries of meetings---the focus of this thesis---is a rarely researched topic. % TODO Maybe briefly explain what abstractive summaries are
Indeed, most of the research on summarizing meetings has been focused on either extractive summarization or unsupervised approaches such as dependency graph fusion \cite{1609.07035} or multi-sentence compression \cite{shang-etal-2018-unsupervised}.

% ==============
% PROBLEM DEFINITION
% ==============

\section{Problem Definition}

When generating abstractive summaries of meetings with deep learning, one usually faces two problems: the small amount of available training data and the very long transcripts of meetings \cite[p.~650--651]{10.1007/978-3-030-20521-8_53}.
The former makes it difficult for a neural network to properly learn as deep learning approaches tend to need a large amount of data to produce good results.
The long transcripts are a problem as modern sequence-to-sequence architectures either struggle to learn long range dependencies \cite{Hochreiter01gradientflow} or have unreasonably high memory requirements for long sequence length.
The goal of this work is therefore to analyze if a data-driven approach is still feasible. % Approach to what
Specifically, BERT should be utilized for transfer learning and thus limit the amount of required training data to achieve good results.
Additionally, a method should be developed to address the issue of long sequence length.
This is especially crucial when using BERT, which was only pretrained with a sequence length of 512 \cite[p.~13]{devlin2018bert}, much shorter than a typical meeting transcript.

% ======
% METHOD
% ======

\section{Method}

As part of this work, a network is developed that utilizes BERT and is fine-tuned for abstractive meeting summarization.
It is trained with data from the two most commonly used meeting corpora, the AMI Meeting Corpus \cite{Mccowan05theami} and the ICSI Meeting Corpus \cite{Janin}, which are described in \cref{ch:data}.
To validate the results, automatic evaluation using Recall-Oriented Understudy for Gisting Evaluation (ROUGE)\footnote{ROUGE is explained in \cref{ssec:automatic-evaluation}.} is performed, and the achieved scores are then compared with scores from other approaches for abstractive meeting summarization.
Additionally, cross-corpus validation\footnote{For example, training on the ICSI corpus and testing on the AMI corpus.} is performed to evaluate if the network can generalize what it learned.

% =================
% THESIS OUTLINE
% =================

\section{Thesis Outline}

\Cref{ch:data} introduces the two meeting corpora used to train the network.
\Cref{ch:theoretical-foundations} explains the theoretical foundations necessary for this work.
\Cref{ch:system-description} describes the model architecture of the network used in this thesis.
It also explains how the network was trained and what steps were performed to prepare the training data.
Additionally, it focuses on the actual implementation with the Texar library \cite{hu2019texar} and what hyperparameters were used for training.
\Cref{ch:experiments} presents the experiments performed on the network and what scores it achieved compared to other summarization approaches and a simple baseline.
It also provides several hand-picked example outputs.
\Cref{ch:outlook} gives an outlook for possible future research.
Finally, \cref{ch:summary-and-conclusion} summarizes the results of the work and draws a conclusion on how they can be interpreted.
