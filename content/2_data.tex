\chapter{Data}\label{ch:data}

For this work, two corpora for meeting data are used, the AMI Meeting Corpus \cite{Mccowan05theami} and the ICSI Meeting Corpus \cite{Janin}.

% ==============
% AMI MEETING CORPUS
% ==============

\section{AMI Meeting Corpus}\label{sec:ami-meeting-corpus}

The AMI Meeting Corpus is a corpus for meetings, published by the AMI (Augmented Multi-party Interaction) project in 2006 \cite{Mccowan05theami}.
It contains very detailed annotation of 100 hours of meeting recordings.
The relevant aspects and annotation of the AMI Meeting Corpus, that are relevant for this work, are briefly explained below.

\subsection{Scenario and Non-Scenario Meetings}

The AMI Meeting Corpus contains data for two different types of meetings: Scenario meetings and non-scenario meetings.
For this work, only the data from the scenario meetings are used, as the non-scenario meetings are missing some annotations that are necessary for this work.
The following paragraphs describe the scenario of the scenario meetings.

For the scenario meetings, participants play employees of an electronics company that work together on a project.
They are part of a design team, which tries to develop a prototype for a television remote, because the existing ones are considered not user-friendly, unattractive and old-fashioned.
The participants are assigned one of the following roles:
\begin{itemize}
\item A project manager
\item A marketing expert
\item A user interface designer
\item An industrial designer
\end{itemize}

Each project consists of 4 meetings:
\begin{itemize}
\item A project kick-off meeting
\item A meeting for the functional design of the remote, such as user requirements, technical functionality and the working design
\item A meeting for the conceptual design of the remote's components, such as the materials or the user interface
\item A final meeting to finalize the design and evaluate the results.
\end{itemize}
Before each meetings, the participants perform individual work.

In total, data for 35 of these projects is available, each with the 4 meetings. \cite[p.~2]{Mccowan05theami}

\subsection{Annotations}\label{ssec:ami-annotations}

The following annotations are available and used in this work:

\paragraph{Dialogue Acts}

The transcription of the whole meeting is split into smaller pieces on a per-person basis.
These subsets of the transcription are called "Dialogue Acts".
A dialogue act tries to group word, that belong together to form a speaker's attention, e.g. a question.
Every word of the transcription is part of exactly one dialogue act.
Each dialogue act is classified by its content, but this classification is not used in this work. \cite{amiWebsite}

\paragraph{Topic Segmentation}

Each meeting is split into multiple topics that by themselves may be split into subtopics.
Examples for topics are "industrial designer presentation", "evaluation of prototype" or "project specs and roles of
participants" to name a few. \cite{amiWebsite}

\paragraph{Abstractive Summaries}

\begin{figure}[h]
\begin{lstlisting}[numbers=none]
S1: The project manager introduced the upcoming project to the 
    team members and then the team members participated in an
    exercise in which they drew their favorite animal and
    discussed what they liked about the animal.
S2: The project manager talked about the project finances and
    selling prices.
S3: The team then discussed various features to consider in making
    the remote.
\end{lstlisting}
\caption{Example abstract of abstractive summary for scenario meeting \texttt{ES2002a}.}
\label{fig:abstractive-summary-example}
\end{figure}

For each meeting, abstractive summaries are available.
Each abstractive summary consists of an abstract, decisions, problems/issues and actions, but this work is only using the abstract.
Usually, the abstract consists of multiple sentences.
An example for such an abstract with 3 sentences is shown in \cref{fig:abstractive-summary-example}.

Additionally, for each sentence of the abstractive summary, one or more dialogue acts are selected, that are linked together.
This mapping of $n$ dialogue acts to $1$ sentence of the summary is later used for training, as described in \label{sec:concept-training} in more detail. \cite{amiWebsite}

\subsection{Segmentation of the Corpus}\label{ssec:ami-segmentation-of-the-corpus}

\begin{table}[h]
\centering
\begin{tabular}{@{}ll@{}}
\toprule
\multicolumn{1}{c}{Set} & \multicolumn{1}{c}{Time} \\ \midrule
Training & $\sim$50 hours \\
Development & $\sim$11 hours \\
Test & $\sim$11 hours \\ \bottomrule
\end{tabular}
\caption[Distribution of meeting time of train, dev and test sets]{Distribution of meeting time of train, dev and test sets \cite{amiWebsite}.}
\label{tab:meeting-time-distribution}
\end{table}

To make results comparable with other works that use the corpus, it is split into training, development and test sets.
For the scenario meetings, the data distribution is shown in \cref{tab:meeting-time-distribution}. \cite{amiWebsite}

% ==============
% ICSI MEETING CORPUS
% ==============

\section{ICSI Meeting Corpus}

The ICSI Meeting Corpus, that contains meetings that were collected at the International Computer Science Institute (ICSI) over a period of three years.
It contains about 72 hours of meeting data. \cite{Janin}
As it has a very similar structure than the AMI Meeting corpus described in \cref{sec:ami-meeting-corpus}, this chapter will only focus on their differences that are relevant for this work.

\paragraph{Scenario}

Unlike the scenario meetings of the AMI Meeting Corpus, the ICSI Meeting Corpus are natural meetings that do not have a specific topic.
However, as the meetings are all recorded at the ICSI, they usually follow a specific type.
Because of this, every meeting of the corpus is classified as on one of 5 different meeting types. \cite{Janin}
% TODO Maybe name/describe the 5 types of meetings. But as they are not really important, I will omit them for now...

\paragraph{Topic Segmentation Annotation}

For ICSI, only automatic topic segmentation is available, but no manual segmentation like for AMI.

\paragraph{Segmentation of the Corpus}

Unlike the AMI corpus, the ICSI Meeting Corpus does not come with a predefined split for training, development and test sets.
Because of this missing split, this work will mainly work with the AMI Meeting Corpus and only use the ICSI Meeting Corpus for cross-corpus-validation.